\section{Conclusies}
\label{sec:conc}

Het onderzoek had als doel het repliceren van een eerder onderzoek, namelijk dat van \cite{state}, op een andere dataset. De dataset die hiervoor gebruikt is bestond uit Nederlandse troonredes van 1814 tot 2014. Hierbij werd vooral gekeken naar de tekstanalyse technieken voor het weergegeven van verschuivingen van onderwerpen binnen corpora door gebruik te maken van co-occurences. 

Er werd allereerst gekeken of de technieken gebruikt binnen \cite{state} daadwerkelijk toepasbaar waren op onze dataset en relevante resultaten leverden. Hiervoor werd een specifieke definitie gesteld wat voor ons als relevant wordt gezien. Aan de hand van de resultaten is gebleken dat de toegepaste technieken wel degelijk resultaten leverden die relevant voor ons waren. Hierdoor hebben we kunnen stellen dat de technieken gebruikt binnen \cite{state} ook op onze dataset gebruikt kon worden.

Na dit vast te stellen is er gekeken naar het verband tussen de resultaten van het uitvoeren van de tekstanalyse technieken op het corpus met de werkelijkheid. Hier werd duidelijk dat de resultaten te herleiden waren en een duidelijke link hebben met de werkelijkheid. Hoewel één moment hier zeer uitspringt is het echter moeilijk om andere momenten net zo exact terug te koppelen. Er zijn echter wel duidelijk trends te zien. Deze trends geven weer welke onderwerpen jaarlijks meer of minder ter sprake kwamen.

Uiteindelijk kunnen we hiermee antwoord geven op de hoofdvraag van dit onderzoek, \textit{"Hoe kan "woord co-occurence" als tekstanalyse techniek worden gebruikt om de verschuiving in onderwerpen/thema’s over 200 jaar troonredes weer te geven?"}

Woord co-occurences kunnen gebruikt worden om de meest belangrijke woorden en meest voorkomende verbanden tussen woorden in een tekst weer te geven. Hierbij moet er wel een duidelijk onderscheid gemaakt worden tussen het soort woorden dat men gebruikt uit de tekst. De meest indicatieve soort woorden hebben wij hier gevonden als de zelfstandige naamwoorden, maar dit kan voor andere corpora mogelijk anders zijn. Uiteindelijk hebben we aan de hand van de veel voorkomende co-occurences termen aan onderwerpen kunnen toewijzen die behandeld werden in het corpus. Doordat elk van deze onderwerpen een eigen lijst met termen had die het onderwerp impliceerde hebben we kunnen weergeven welke onderwerpen elk jaar het meest aan bod kwamen en hoe de behandeling van de onderwerpen verdeeld was.  

Uiteindelijk is er een duidelijk overzicht gevormd van de verschuiving van onderwerpen over de jaren in figuur \ref{onderwerpverdeling}. De resultaten weergegeven in deze afbeelding zijn teruggekoppeld naar de werkelijkheid waaruit gesteld kan worden dat de resultaten valide zijn. 

Voor vervolg onderzoek zou gekeken kunnen worden naar het gebruik van deze technieken op andere corpora. Ook zou er gekeken kunnen worden naar het opstellen van een model waar randvoorwaarden in staan waar teksten binnen een dataset aan moeten voldoen om met deze methoden onderzocht te worden. Dit zouden andere modellen kunnen zijn voor verschillende vakgebieden, omdat daar mogelijk niet alleen gekeken moet worden naar de zelfstandige naamwoorden in de teksten. Vanuit zo'n model zou ook duidelijk opgesteld kunnen worden wat voor uitspraken gedaan kunnen worden aan de hand van de resultaten.


\subsection{Acknowledgements}
