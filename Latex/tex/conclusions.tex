\section{Conclusions}
\label{sec:conc}

Het onderzoek had als doel het repliceren van een eerder onderzoek op een andere dataset. De dataset die hiervoor gebruikt is bestond uit Nederlandse troonredes van 1814 tot 2014. Hierbij werd vooral gekeken naar de tekstanalyse technieken voor het weergegeven van verschuivingen van onderwerpen binnen corpora door middel van co-occurences. 

Er werd allereerst gekeken of de technieken gebruikt binnen \citep{state} daadwerkelijk toepasbaar waren op onze dataset en relevante resultaten leverden. Hiervoor werd een specifieke definitie gesteld wat voor ons als relevant wordt gezien. Aan de hand van de resultaten is gebleken dat de toegepaste technieken wel degelijk resultaten leverden die relevant voor ons waren. Hierdoor hebben we kunnen stellen dat de technieken gebruikt binnen \citep{state} ook op onze dataset gebruikt kon worden.

Na dit vast te stellen is er gekeken naar het verband tussen de resultaten van het uitvoeren van de tekstanalyse technieken op het corpus met de werkelijkheid. Hier werd duidelijk dat de resultaten te herleiden waren en een duidelijke link hebben met de werkelijkheid. Hoewel één moment hier zeer uitspringt is het echter moeilijk om andere momenten net zo'n exact terugkoppeling te geven. Er zijn echter wel duidelijk trends te zien. Deze trends geven weer welke onderwerpen jaarlijks meer of minder ter sprake kwamen.


\subsection{Acknowledgements}
