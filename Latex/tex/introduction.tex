\section{Introduction}
\label{sec:intro}

Dit onderzoek is een replicatie-onderzoek aan de hand van een eerder uitgevoerd onderzoek dat als paper is gepubliceerd. Het paper in kwestie beschrijft een onderzoek en de resultaten hiervan naar de "State of the Union" speeches uit Amerika van \cite{state}. In dat onderzoek werd gekeken naar wat er veranderde binnen de speeches sinds ze voor het eerst werden gehouden en hoe de speeches er hedendaags uitzien. Er werd gekeken naar de verschuiving in taalgebruik en inhoud in de speeches. Dit werd gedaan om een beeld te krijgen van wat de belangrijke onderwerpen in een speech waren en welke thema's behandeld werden over de jaren. Hieruit werd gekeken naar het moment dat de verschuiving naar het moderne Amerikaanse politieke bewustzijn plaatsvond. In het onderzoek werd gekeken naar de speeches van 1790 t/m 2014 welke jaarlijks uitgesproken worden door de president.

De State of the Union is niet de enige speech die in deze vorm al lange tijd jaarlijks wordt gehouden. Zo ook de Nederlandse troonredes. Deze troonredes worden sinds 1814 jaarlijks door de koning(in) voorgedragen. De troonredes kunnen gezien worden als een Nederlandse versie van de State of the Union. De troonredes bevatten echter ook nog vaak reflecties op het afgelopen jaar naast van een samenvatting over de staat van het land. De staat van het land wordt ook besproken maar dit is meer gericht op een kijk naar de positie van Nederland in de wereld. Dit is echter niet altijd het geval geweest. Zo was de inhoud vroeger anders dan nu, net als het taalgebruik.

Tijdens het onderzoek zal er gekeken worden naar de veranderingen in de troonredes over de jaren. Het gaat hier vooral om de onderwerpen en thema's die behandeld worden. Dit zal onderzocht worden aan de hand van automatische tekstanalyse technieken. Dit om ervoor te zorgen dat er geen handwerk aan te pas komt, zodat alles controleerbaar is en zoveel mogelijk menselijke fouten worden voorkomen. Als deze technieken goed werken zouden ze toegepast kunnen worden op andere corpora om een snel overzicht te geven van de inhoud of verschuiving van inhoud over tijd van deze corpora. Er zal allereerst niet specifiek worden gekeken naar de verandering in taalgebruik en grammatica, maar dit kan wel gebruikt worden om een beeld te vormen over de veranderingen binnen de troonredes.

De terugkoppeling naar het informatiekunde vakgebied ligt hier bij het automatiseren van verschillende technieken en het weergeven en gebruik van informatie. Doordat deze technieken worden geautomatiseerd kunnen ze mogelijk op vele verschillende plekken worden toegepast. Ook kunnen er betere controles worden uitgevoerd omdat alles uiteindelijk door computers wordt uitgevoerd. Verder wordt er gekeken naar de mogelijkheid om de technieken toe te passen op andere datasets en andere toepassingen van de technieken. De informatie kan op verschillende manieren worden gebruikt en weergegeven. Hierbij zullen we kijken naar de manier waarop informatie wordt gerepresenteerd en wat men vanuit deze representatie met de informatie kan doen.  

Om het onderzoek een duidelijke richting te geven zal gepoogd worden de volgende onderzoeksvraag te beantwoorden:

\begin{itemize}
\item Hoe kan "word co-occurence" als tekstanalyse techniek worden gebruikt om de verschuiving in onderwerpen/thema's over 200 jaar troonredes weer te geven?
\end{itemize}
We   beantwoorden deze vraag met behulp van de volgende deelvragen:
\begin{enumerate}
\item Geven de methodes gebruikt in \cite{state} relevante resultaten over het Nederlandse troonrede corpus?
\item Hoe komen de resultaten overheen met de werkelijkheid?

%\paragraph{Overview of thesis}

