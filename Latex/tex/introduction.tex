\section{Introduction}
\label{sec:intro}

Dit onderzoek is een replicatie-onderzoek aan de hand van een eerder uitgevoerd onderzoek dat als paper is gepubliceerd. Het paper in kwestie beschrijft een onderzoek naar de "State of the Union" speeches uit Amerika ~\cite{state}. In dat onderzoek werd gekeken naar wat er veranderde binnen de speeches sinds ze voor het eerst werden gehouden en hoe de speeches er hedendaags uitzien. Er werd gekeken naar de verschuiving in taalgebruik en inhoud in de speeches. Dit werd gedaan om een beeld te krijgen wat de belangrijke onderwerpen in een speech waren en welke thema's behandelt werden over de jaren.
\\
De State of the Union is niet de enige speech die al lange tijd wordt gehouden. Zo ook de troonredes van ons eigen Nederland. Deze troonredes worden sinds 1818 jaarlijks door de koning/koningin gegeven. De troonredes kunnen gezien worden als een Nederlandse versie van de State of the Union. De troonredes bevatten echter vaak reflecties op het afgelopen jaar in plaats van een update over de staat van het land. De staat van het land wordt ook wel besproken, maar meer in de context van de wereld, namelijk een kijk naar de positie van Nederland in de wereld. Dit is echter niet altijd het geval geweest. Zo was de inhoud vroeger anders dan nu, net als het taalgebruik.
\\
Tijdens het onderzoek zal er gekeken worden naar de veranderingen in de troonredes over de jaren. Het gaat hier vooral om de onderwerpen en thema's die behandeld worden. Er zal allereerst niet specifiek worden gekeken naar de verandering in taalgebruik en grammatica, maar dit kan wel gebruikt worden om een beeld te vormen over de veranderingen van de troonredes.
\\
Om het onderzoek een duidelijke richting te geven zal gepoogd worden de volgende onderzoeksvraag te beantwoorden:

\begin{itemize}
\item Hoe kan "word co-occurence" als tekstanalyse techniek worden gebruikt om de verschuiving in onderwerpen/thema's over 200 jaar troonredes weer te geven?
\end{itemize}

\begin{description}
\item[RQ1]  We   beantwoorden deze vraag met behulp van de volgende deelvragen:
\begin{enumerate}
\item Wat is "word co-occurence" en hoe worden deze uit teksten verzameld? Evaluatie sectie~\ref{sec:eva}.
\item Hoe kan betekenis worden gegeven aan co-occurences?
\item Hoe wordt de context van de tekst in acht gehouden?
\item Wat voor invloed heeft de verandering van taal(gebruik) over de jaren op de analyse technieken?
\end{enumerate}
\end{description}

De dataset die gebruikt gaat worden om deze vragen te beantwoorden is een verzameling van troonredes sinds 1814. Deze troonredes zijn terug te vinden op www.troonredes.nl~\cite{troonredes} .  Om een duidelijker beeld te geven van wat de troonrede nu precies is eerst een kleine introductie van wat de troonredes nu precies zijn.

De troonrede wordt jaarlijks door de koning(in) uitgesproken op Prinsjesdag. De eerste troonrede werd in 1814 als een algehele toespraak voor de Staten-generaal gehouden. De troonredes worden vooral gebruikt om wets- en beleidsveranderingen door te geven en als beschouwing op het afgelopen jaar en de staat van het land. In recentere jaren heeft deze beschouwing zich ook uitgebreid naar gebeurtenissen door de hele wereld die invloed uitoefenen op de Nederlandse staat.

Er is niet elk jaar een troonrede gegeven. Dit heeft verschillende redenen, zoals oorlogen, angst voor rellen, onvrede over het kabinet of gezondheidsredenen omtrent de koning(in). Dit zorgt ervoor dat niet elk jaar sinds 1814 wordt gerepresenteerd door een troonrede. Sinds 1848 worden de troonredes geschreven door ministers en is het kabinet verantwoordelijk voor de uitspraken. Dit zorgt ervoor dat de troonredes een beeld geven van wat de Nederlandse regering op dat moment belangrijk vindt.


\paragraph{Overview of thesis}
Hier geef je even kort weer wat in elke sectie staat.