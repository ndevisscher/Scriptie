\section{Introduction}
\label{sec:intro}

\todo{Dit is een prima introductie. Goed gedaan! Nog wat puntjes: geef duidelijk aan dat het je gaat om **volledig automatische technieken**, er wordt dus niks met de hand gedaan.\\
probeer het verband met Informatiekunde in minstens 1 paragraaf uit te werken. Dat zal de 2e lezer graag willen zien. En ik ook. Waarom kan je dit onderzoek doen binnen informatiekunde? Wat is er specifiek aan? Kan je wat voorbeelden geven van andere corpora die je ook zo zou kunnen aanpakken, en wat dus in korte tijd een mooi overzicht geeft over dat corpus.}

Dit onderzoek is een replicatie-onderzoek aan de hand van een eerder uitgevoerd onderzoek dat als paper is gepubliceerd. Het paper in kwestie beschrijft een onderzoek naar de "State of the Union" speeches uit Amerika van \cite{state}. In dat onderzoek werd gekeken naar wat er veranderde binnen de speeches sinds ze voor het eerst werden gehouden en hoe de speeches er hedendaags uitzien. Er werd gekeken naar de verschuiving in taalgebruik en inhoud in de speeches. Dit werd gedaan om een beeld te krijgen van wat de belangrijke onderwerpen in een speech waren en welke thema's behandeld werden over de jaren. Hieruit werd gekeken naar het moment dat de verschuiving naar het moderne Amerikaanse politieke bewustzijn plaatsvond. In het onderzoek werd gekeken naar de speeches van 1790 t/m 2014 welke jaarlijks uitgesproken worden door de president.

De State of the Union is niet de enige speech die al lange tijd jaarlijks wordt gehouden. Zo ook de Nederlandse troonredes. Deze troonredes worden sinds 1814 jaarlijks door de koning(in) voorgedragen. De troonredes kunnen gezien worden als een Nederlandse versie van de State of the Union. De troonredes bevatten echter vaak reflecties op het afgelopen jaar in plaats van een samenvatting over de staat van het land. De staat van het land wordt ook besproken maar dit is meer gericht op een kijk naar de positie van Nederland in de wereld. Dit is echter niet altijd het geval geweest. Zo was de inhoud vroeger anders dan nu, net als het taalgebruik.

Tijdens het onderzoek zal er gekeken worden naar de veranderingen in de troonredes over de jaren. Het gaat hier vooral om de onderwerpen en thema's die behandeld worden. Er zal allereerst niet specifiek worden gekeken naar de verandering in taalgebruik en grammatica, maar dit kan wel gebruikt worden om een beeld te vormen over de veranderingen van de troonredes.

Om het onderzoek een duidelijke richting te geven zal gepoogd worden de volgende onderzoeksvraag te beantwoorden:

\begin{itemize}
\item Hoe kan "word co-occurence" als tekstanalyse techniek worden gebruikt om de verschuiving in onderwerpen/thema's over 200 jaar troonredes weer te geven?
\end{itemize}
We   beantwoorden deze vraag met behulp van de volgende deelvragen:
\begin{enumerate}
\item Geven de methodes gebruikt in \cite{state} relevante resultaten over het Nederlandse troonrede corpus?
\item Hoe komen de resultaten overheen met de werkelijkheid?

%\paragraph{Overview of thesis}

