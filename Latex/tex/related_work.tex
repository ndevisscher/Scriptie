\section{Related Work}
\label{sec:rel}

\subsection{Latent Dirichlet Allocation (LDA)}
LDA is een statistisch model dat uitgaat van observaties. Hierbij kunnen groepen van observaties worden uitgelegd door on-geobserveerde groepen. Dit gebeurt doordat de on-geobserveerde groepen uitleggen waarom sommige delen van de data gelijk zijn aan de geobserveerde data. Binnen de tekstanalyse kan dit als volgt worden gezien: Observaties zijn woorden verzameld in documenten. Hierbij wordt gesteld dat elk document een mix is van een klein aantal onderwerpen en het gebruik van een woord binnen de tekst is toe te weiden aan één van de onderwerpen van het document. Elk woord vloeit dus voort uit een gelimiteerd aantal onderliggende onderwerpen, terwijl er een oneindige hoeveelheid van onderwerpen kunnen zijn. 

Voor tekstanalyse kan dit worden gebruikt om een beeld te vormen van de verschillende onderwerpen die een tekst mogelijk representeert. Er moet hier wel rekening gehouden worden dat dit geen exacte indicatie is van de onderwerpen. Dit doordat er geen rekening wordt gehouden met de context van de woorden, maar enkel met de onderwerpen waar de woorden aan toe te wijzen zijn. \citep{blei2003latent}

\subsection{Latent Semantic Analysis (LSA)}
LSA is een techniek vanuit de natuurlijke taalverwerking. De techniek analyseert de relatie tussen groepen van documenten en de termen die deze bevatten. Dit wordt gedaan door te kijken naar concepten die de relatie tussen de documenten en termen weer kunnen geven. LSA is gebaseerd op een stelling. Deze stelling is dat woorden die gelijksoortige betekenissen hebben, zoals "huis" \& "woning", in soortgelijke delen van tekst voorkomen. LSA maakt gebruik van een matrix waarin bijgehouden wordt hoe vaak een woord voorkomt per paragraaf. Deze matrix wordt gereduceerd waarbij de gelijkheid tussen paragrafen behouden wordt door middel van singulierewaardenontbinding(SWO). SWO wordt hier gebruikt om de matrix te ontbinden naar de relevante kolommen, welke representatief zijn voor soortgelijke paragrafen. 

Na het reduceren worden de woorden uit het matrix vergeleken door de cosinus van de vector van hun rij te nemen. Dit geeft een waarde van 0 tot 1, waarbij een waarde van 1 wil zeggen dat de woorden erg gelijk aan elkaar zijn en een waarde van 0 aangeeft dat de woorden zeer ongelijk aan elkaar zijn. \citep{dumais2004latent}

\subsection{Topic Modelling}
Topic modelling is een type statistisch model voor het vinden van abstracte "topics", ook wel onderwerpen, in collecties van documenten. Binnen de tekstanalyse wordt topic modelling vooral gebruikt om verborgen achterliggende semantische structuren tussen documenten weer te geven. Dit wordt gedaan door te kijken naar de samenhang tussen onderwerpen binnen de verschillende teksten. Deze onderwerpen die uit de topic modelling technieken voortkomen zijn uiteindelijk gevormd vanuit clusters van gelijksoortige woorden.

Topic modelling geeft idealiter ook een indruk van het taalgebruik dat men kan verwachten binnen een tekst. Zo verwacht men dat binnen een tekst over voetbal de woorden "keeper" \& "buitenspel" vaker voorkomen dan bijvoorbeeld het woord "basketbal". Omdat een tekst echter meerdere onderwerpen kan behandelen kan er worden gekeken naar de proporties van de onderwerpen binnen de tekst. Zo kan een tekst over sport 80\% voetbal zijn en 20\% basketbal. Men zou hier dan verwachten dat er ongeveer 4 keer zoveel voetbal gerelateerde woorden in de tekst voorkomen dan dat er basketbal gerelateerde termen voorkomen. \citep{sojka2010software}

\subsection{Digital Humanities}
Digital Humanities, digitale geesteswetenschappen, is het onderzoeksgebied dat zich bezighoud met de kruising tussen computerwetenschappen en geesteswetenschappen. Hier wordt onderzoek gedaan aan de hand van gedigitaliseerd materiaal en materiaal dat digitaal is gemaakt. Het bestrijkt verschillende onderwerpen, waaronder data mining, data organisatie en informatie verzameling. Hierbij gaat het veelal om grote datasets welke vaak online te vinden zijn. 

De definitie van digital humanities wordt constant aangepast, omdat er steeds meer onderzoek naar en binnen gedaan wordt. Ook wordt het steeds duidelijker dat digital humanities een steeds groter gebied bestrijkt. Dit doordat er steeds meer onderzoek wordt gedaan naar de mogelijkheden van het gebruik van computers om taken te automatiseren. Ook kunnen computers steeds meer taken uitvoeren waardoor er meer digitaal uitgevoerd kan worden.\citep{berry2012understanding}

