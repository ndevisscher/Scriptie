\section{Related Work}
\label{sec:rel}

Deze sectie bestaat uit een aantal "blokken", waarin je per blok de relevante literatuur beschrijft. Neem alleen literatuur op die van belang is voor jouw onderzoeksvraag en deelvragen.Typisch heb je 1 blok voor je hoofdvraag en per deelvraag \textbf{RQi} een blok. 

\subsection{Het gebruik van co-occurrences in tekst analyse}

\subsubsection{co-occurrence extractie}
Men kan op verschillende manieren co-occurrences uit teksten halen. Om deze manieren te verduidelijken gebruiken we de volgende voorbeeldzin: "Het is erg heet". Vanuit deze zin kunnen op de volgende manieren co-occurences worden gehaald. Door enkel co-occurrences te gebruiken van woorden die exact naast elkaar in een zin staan, "Het,is" \& "is,erg" \& "erg,heet". Ook kan het door woorden in een zin volgens de zinsvolgorde met elkaar te koppelen zelfs als er andere woorden tussen, hierdoor zouden "Het,heet" \& "Het,erg" \& "is,heet" ook co-occurrences zijn. Of het kan door alle mogelijke combinaties van woorden in een zin te vormen, dit gebeurt veelal op alfabetische woordvolgorde. Volgens de laatste manier wordt geen rekening meer gehouden met de volgorde van de woorden. De manier die het beste is om de co-occurrences uit de tekst te halen is afhankelijk van wat men uiteindelijk wil kunnen zeggen met de co-occurrences. \citep{shimohata1997retrieving}

\subsubsection{relevantie van co-occurrences}
Co-occurrences zijn uiteindelijk simpele combinaties van woorden. Deze combinaties geven aan of twee woorden samen voorkomen in een zin, paragraaf of tekst. Hierbij wordt meestal gekeken naar het samen voorkomen in een zin. Met behulp van co-occurrences kan men verschillende clustering methodes uitvoeren. De clusters gevormd door deze methodes kunnen weergeven welke woorden veel invloed hebben op de tekst doordat deze het middelpunt van clusters zullen zijn. Door middel van deze clusters kunnen de meest belangrijke termen dus worden afgeleid. Ook kunnen met behulp van co-occurrences veel voorkomende multi termen worden gevonden in teksten. Enkele voorbeelden hiervan zijn: "Verenigde Staten" en "Hoger onderwijs". Deze multi termen zijn individuele woorden die ook als één woord gezien kunnen worden. Met behulp van het bepalen van co-occurrences kunnen deze multi termen makkelijk gevonden worden. 

\subsection{Induceren van het onderwerp van een tekst}
\subsubsection{Trefwoord extractie}
Er bestaan verschillende manieren om trefwoorden uit teksten te halen. De meeste eenvoudige manier is door simpelweg te kijken naar de waarschijnlijkheid dat een woord voorkomt. Dit wordt gedaan door te kijken naar de frequentie dat een woord voorkomt in een tekst en dit te delen door het totaal aantal woorden in de tekst. Verder kan men kijken naar een geheel corpus, de meest bekende methode hiervoor is TF-IDF (term frequency–inverse document frequency)\citep{ramos2003using}. Hierbij wordt gekeken naar het voorkomen van woorden in een tekst tegenover het voorkomen in het gehele corpus. De TF-IDF score van een woord is hoog voor een specifieke tekst uit een corpus als deze vaak voorkomt in die tekst, maar verder weinig in de rest van het corpus. Door de frequentie van het voorkomen van een woord in een tekst te compenseren met het voorkomen van het woord in de gehele tekst wordt rekening gehouden met woorden die in het algemeen veel worden gebruikt, zoals stopwoorden \citep{aggarwal2012survey}. 
Daarnaast zijn er methodes gericht op individuele teksten zoals in \cite{matsuo2004keyword}. Hier wordt een algoritme gebruikt om trefwoorden uit een enkele tekst te halen zonder gebruik te maken van een corpus. 

\subsubsection{Invloed van context}
Om een onderwerp of meerdere onderwerpen toe te kunnen wijzen aan een tekst moet men rekening houden met meerdere aspecten. De meest belangrijke hiervan is de context van de tekst. De context bepaald namelijk op welke manier woorden geïnterpreteerd worden en wat ze betekenen voor de tekst. Zo zijn er woorden die enkel voor specifieke domeinen betekenis hebben. Er zijn al meerdere methodes ontwikkeld om rekening te houden met de context van een tekst, maar de meeste hiervan hebben hiervoor een corpus nodig om het context netwerk te kunnen bouwen. Om van een individuele tekst de context te kunnen bepalen kan zeer lastig zijn. Er kan rekening gehouden worden met de context van een corpus door gebruik te maken van woordenboeken voor het specifieke domein dat het corpus bestrijkt. Hiervoor moet men echter wel zelf bepalen wat het domein van het corpus is. \citep{aggarwal2012survey}